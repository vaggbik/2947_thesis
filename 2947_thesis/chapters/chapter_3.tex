\section{Αλγόριθμοι Βέλτιστης Διαδρομής (ή βασικοί αλγόριθμοι γράφων)}

Ανάλυση και επεξήγηση βασικών αλγορίθμων γράφων, όπως η αναζήτηση πλάτους πρώτης (BFS), η αναζήτηση βάθους πρώτης (DFS), ο αλγόριθμος της Dijkstra για τον ελάχιστο μονοπάτι, ο αλγόριθμος του Kruskal για το ελάχιστο συνεκτικό δέντρο κ.λπ.

Θα παρουσιαστούν ορισμένοι αλγόριθμοι που θα επιλέξω και θα γραφούν αναλογα υποπαράγραφοι για τον καθένα

\subsection{Εισαγωγή στους Αλγόριθμους Βέλτιστης Διαδρομής} 
Η εικασία των 4 χρωμάτων πρόβλημα του χρωματισμού χαρτών, αναπαράσταση προσώπων μέσω σημείων και ένωση των σημείων στις περιπτώσεις ύπαρξης προσωπικών σχέσεων, αναπαράσταση μοριακών δομών και αλληλεπιδράσεων τους
Παρουσίαση πρακτικών εφαρμογών γράφων, όπως κοινωνικά δίκτυα, μεταφορές, δρομολόγηση, χαρτογραφία, προβλήματα προσβασιμότητας κ.λπ. Επίδραση της δομής του γράφου στην επίλυση προβλημάτων.

\subsection{Οι πιο γνωστοί Αλγόριθμοι Βέλτιστης Διαδρομής}
\selectlanguage{english}
\subsubsection{Dijkstra}
\subsubsection{A*}
\subsubsection{Floyd-Warshall}
\subsubsection{Αλγόριθμος Αποικίας Μυρμηγκιών}
μη ξεχασεις να κανεις αναφορα στη φερομονη.

\subsection{Σύγκριση Α.Α.Μ. με υπόλοιπους}
%%%%%%%%%%%%
ACO [1, 24] is a class of algorithms, whose first member, called Ant System, wasinitially proposed by Colorni, Dorigo and Maniezzo [13, 21, 18]. The main underlying idea, loosely inspired by the behavior of real ants, is that of a parallel search over several constructive computational threads based on local problem data and on a dynamic memory structure containing information on the quality of previously obtained result. The collective behavior emerging from the interaction of the different search threads has proved effective in solving combinatorial optimization (CO) problems. Οι μετευρετικοί αλγόριθμοι συνήθως εμπνέονται από φυσικά φαινόμενα, όπως η επιλογή φυσικής εξέλιξης ή η συμπεριφορά των ζώων, και εφαρμόζουν αυτές τις ιδέες στο πρόβλημα που προσπαθούν να λύσουν. Συχνά, οι μετευρετικοί αλγόριθμοι είναι ικανοί να εξερευνήσουν μεγάλους χώρους αναζήτησης και να βρουν καλές λύσεις σε προβλήματα που θεωρούνται δύσκολα για άλλες μέθοδους


2.2 Local search \cite{Dorigo-Stutzle}
