\section{Εισαγωγή}
\selectlanguage{english}
Ένα μεμονωμένο μυρμήγκι δεν μπορεί να κάνει πολλά, τα μυρμήγκια ως σύνολο όμως, όπως και κάθε άλλο είδος εντόμων που λειτουργούν ομαδικά ως σμήνος για την επίτευξη κοινού στόχου, μπορούν να ολοκληρώσουν εξειδικευμένες εργασίες γρήγορα και αποτελεσματικά. Αυτό αποτελεί πηγή έμπνευσης για τον αλγόριθμο που θα μελετήσουμε σε αυτήν την εργασία καθώς και για πολλούς άλλους αλγόριθμους σμήνους που έχουν ως στόχο την επίλυση πολύπλοκων προβλημάτων βελτιστοποίησης πραγματικού κόσμου. \cite{dorigo2004ant} 

Τα τελευταία χρόνια, τέτοια προβλήματα παρουσιάζονται σε διάφορους κλάδους με πληθώρα θεωρητικών και πρακτικών εφαρμογών όπως το πρόβλημα γεφυρών και το πρόβλημα του πλανόδιου πωλητή (Travelling Salesman Problem - TSP) καθώς και σε κοινωνικά δίκτυα, δίκτυα δρομολόγησης \cite{gkertsis2023thewria}, προβλήματα προσβασιμότητας, βελτιστοποίηση δικτύων επικοινωνίας, σχεδίαση ηλεκτρικών κυκλωμάτων, και άλλα, για το λόγο αυτό, ιδιαίτερο ενδιαφέρον έχει αναπτυχθεί για την μελέτη τους. \cite{manwlopoulos2014thewria}

Ο αλγόριθμος της αποικίας μυρμηγκιών (Ant Colony Optimisation - ACO) που θα αναλυθεί στην παρούσα πτυχιακή εργασία είναι ένας αλγόριθμος νοημοσύνης σμήνους και με χρήση αυτού αντιμετωπίζονται τέτοια προβλήματα. Ο όρος "Νοημοσύνη Σμήνους" (swarm intelligence) πρωτοεμφανίστηκε το 1989 από τους Gerardo Beni και Jing Wang \cite{10.1007/978-3-642-58069-7_38} και μας βοηθάνε στην επίλυση προβλημάτων βελτιστοποίησης, ενώ το 1992 προτάθηκε ο ACO μέσω της διδακτορικής διατριβής του Marco Dorigo. Οι αλγόριθμοι αυτοί μιμούνται εξελικτικές διαδικασίες που παρουσιάζονται στην φύση όπως η επιλογή μιας διαδρομής και η προσαρμογή σε νέα δεδομένα. Ο αλγόριθμος της αποικίας των μυρμηγκιών συγκεκριμένα βασίζεται, όπως προκύπτει κι από το όνομα του, στη συμπεριφορά των μυρμηγκιών στην φύση και το πώς καταφέρνουν πάντα να βρίσκουν την βέλτιστη διαδρομή σε όποιο πρόβλημα τους παρουσιαστεί με χρήση πιθανολογικών τεχνικών. Πέρα από τον ACO, παραδείγματα άλλων τέτοιων αλγορίθμων είναι ο αλγόριθμος βελτιστοποίησης σμήνους σωματιδίων (Particle Swarm Optimisation - PSO), ο αλγόριθμος των μελισσών (Artificial Bee Colony Algorithm - ABC), οι διάφοροι γενετικοί αλγόριθμοι (Genetic Algorithms- GA) και πολλοί άλλοι. \cite{mavrovouniotis2017survey}

Ένα κεφάλαιο απαραίτητο για την υλοποίηση αυτού του αλγόριθμου είναι η θεωρία γραφών. Έννοιες στις οποίες βασίζεται ο ACO όπως η φερομόνης και το κόστος διαδρομής μοντελοποιούνται με χρήση αυτής. Κάθε ακμή του γράφου αντιστοιχεί σε μία τιμή φερομόνης, η οποία ανανεώνεται καθώς τα μυρμήγκια περνούν από αυτήν. Τα μυρμήγκια εξερευνούν πιθανές διαδρομές, και ενισχύουν την ποσότητα φερομόνης στις καλύτερες, επηρεάζοντας έτσι τις επιλογές των μυρμηγκιών που ακολουθούν δίνοντας στον αλγόριθμο την δυνατότητα να "θυμάται" προηγούμενες λύσεις και να τις αξιοποιεί στο μέλλον. 


\begin{flushright} 
    I am lost! Where is the line?!
    
    —A Bug’s Life, Walt Disney, 1998\cite{dorigo2004ant}
\end{flushright}
