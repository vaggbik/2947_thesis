\section{Εισαγωγή}
\selectlanguage{english}
Η αποικεία των μυρμηγκιών όπως και κάθε άλλο είδος εντόμων που λειτουργούν ομαδικά ως σμήνος για την επίτευξη κοινού στόχου, μπορούν να ολοκληρώσουν εργασίες τόσο εξιδεικευμένες που είναι απείθανο να το κατάφερνε ένα μεμονωμένο μυρμήγκι. Αυτό αποτελεί πηγή έμπνευσης για τον αλγόριθμο που θα μελετήσουμε σε αυτήν την εργασία καθώς και για πολλούς άλλους αλγόριθμους σμήνους που έχουν ως στόχο την επίλυση πολύπλοκων προβλημάτων βελτιστοποίησης πραγματικού κόσμου. \cite{Dorigo-Stützle2}

Με χρήση του αλγόριθμου της αποικίας μυρμηγκιών (Ant Colony Optimization - ACO) αντιμετωπίζονται διάφορα προβλήματα του πραγματικού μας κόσμο, από την εύρεση βέλτιστης διαδρομή σε χάρτες μέχρι τη βελτιστοποίηση των δικτύων επικοινωνίας. Βασίζεται στη συμπεριφορά των μυρμηγκιών στην φύση και το πώς καταφέρνουν πάντα να βρίσκουν την βέλτιστη διαδομή σε όποιο πρόβλημα τους παρουσιαστεί.

Η θεωρία γραφών είναι απαραίτητη για την υλοποίηση αυτού του αλγορίθμου καθώς ο αλγόριθμος της αποικείας των μυρμηγκιών βασίζεται στην έννοια της φερομόνης και της ευρετικής συνάρτησης που σε έναν υπολογιστή μοντελοποιούνται με χρήση θεωρίας γράφων. Κάθε ακμή του γράφου αντιστοιχεί σε μία τιμή φερομόνης, η οποία ανανεώνεται καθώς τα μυρμήγκια περνούν από αυτήν. Τα μυρμήγκια εξερευνούν πιθανές διαδρομές, και ενισχύουν την ποσότητα φερομόνης στις καλύτερες, επηρεάζοντας έτσι τις επιλογές των μυρμηγκιών που ακολουθούν δίνοντας στον αλγόριθμο την δυνατότητα να "θυμάται" προηγούμενες λύσεις και να τις αξιοποιεί στο μέλλον.

Η εφαρμογή του εκτείνεται από προβλήματα θεωρίας γραφών μέχρι πρακτικές εφαρμογές όπως το πρόβλημα γεφυρών, το πρόβλημα του πλανόδιου πωλητή (Traveling Salesman Problem - TSP) και πολλά άλλα. Πέρα από τη θεωρητική εφαρμογή του αλγόριθμου, έχει εφαρμοστεί επιτυχώς σε πραγματικά προβλήματα όπως η βελτιστοποίηση δικτύων επικοινωνίας, η σχεδίαση ηλεκτρικών κυκλωμάτων, και άλλα.


\begin{flushright}
    I am lost! Where is the line?!
    
    —A Bug’s Life, Walt Disney, 1998
\end{flushright}



