\section{Θεωρία Γράφων}

\subsection{Ιστορική Αναδρομή}
\begin{figure}[ht]
    \begin{minipage}[c]{.46\linewidth}
        \centering
        \includegraphics[scale=0.15]{2947_thesis/pictures/konigsberg.png}
        \caption{Προσομοίωση Κόνιγκσμπεργκ.}
    \end{minipage}
    \hfill%
    \begin{minipage}[c]{.46\linewidth}
        \centering
        \includegraphics[scale=0.15]{2947_thesis/pictures/konigsbergEx.png} 
        \caption{Παράδειγμα διάσχισης γεφυρών}
    \end{minipage}
\end{figure}
\begin{figure}[ht]
    \begin{minipage}[c]{.46\linewidth}
        \centering
        \includegraphics[scale=0.15]{2947_thesis/pictures/konigsberGraph.png}
        \caption{Κόνιγκσμπεργκ ως γράφος.}
    \end{minipage}
    \hfill%
    \begin{minipage}[c]{.46\linewidth}
        \centering
        \includegraphics[scale=0.15]{2947_thesis/pictures/konigsbergEuler.png}
        \caption{Διαδρομή Όυλερ.}
    \end{minipage}
\end{figure}
Η ανάπτυξη της θεωρίας γράφων ξεκίνησε τον 18ο αιώνα και πιο συγκεκριμένα το 1736 στην πόλη \selectlanguage{english} Königsberg της Πρωσσίας. Σήμερα είναι το Ρωσικό \selectlanguage{english} Kaliningrad (μεταξύ Λιθουανίας και Πολωνίας στη Βαλτική) \cite{Μανωλόπουλος}. Η πόλη ήταν χωρισμένη σε 4 τμήματα από τον ποταμό \selectlanguage{english} Pregel \selectlanguage{greek}και χρησιμοποιούνταν 7 γέφυρες για να γίνεται εφικτή η διέλευση των κατοίκων στα διάφορα τμήματά της [Σχήμα 1]. Όταν ο Ελβετός μαθηματικός Λέονχαρντ Όυλερ αναρωτήθηκε αν είναι εφικτό να διασχίσει κάποιος τις γέφυρες της πόλης με βασικό περιορισμό να διασχιστούν όλες οι γέφυρες μόνο μία φορά (Πρόβλημα των γεφυρών του Κόνιγκσμπεργκ) ένας νέος κλάδος των διακριτών μαθηματικών γεννήθηκε, γνωστός και ως θεωρία γράφων. Ο Όυλερ απέδειξε ότι δεν υπάρχει τέτοια διαδρομή μέσω της χρήσης γράφων και κατά συνέπεια το πρόβλημα δεν έχει λύση [Σχήμα 2]. Αυτή η απόδειξη απέκτισε αξία όταν ο Ούλερ την εφάρμοσε και σε άλλα προβλήματα γράφων και γενίκευσε την βασική ιδέα. 

Μια διαδρομή ονομάζεται διαδρομή Ούλερ όταν μπορούμε να επισκεφτούμε κάθε περιοχή-κορυφή διασχίζοντας την κάθε γέφυρα-ακμή μόνο μία φορά (και ονομάζεται κυκλική αν καταλλήγουμε εκεί που ξεκινήσαμε), αν υπάρχει μια τέτοια διαδρομή σε ένα γράφο τότε αυτός ο γράφος ονομάζεται γράφος Όυλερ. \cite{Ντενισιώτης} Στο σχήμα [Σχήμα 3] που αντιπροσοπεύει την πόλη του Κόνιγκσμπεργκ σε μορφή γράφου δεν υπάρχει μια τέτοια διαδρομή. Για να γίνει αυτό πρέπει να αφαιρέσουμε μία γέφυρα-ακμή [Σχήμα 4]. Παρατηρήθηκε από τον Όυλερ ότι όλες οι κορυφές πρέπει να έχουν άρτιο βαθμό εκτός από αυτές που ξεκινά και τελειώνει η διαδρομή, εκτός κι αν η διαδρομή είναι κυκλική. 


\subsection{Εισαγωγή στην Θεωρία Γράφων}

Στην ουσία ένας γράφος είναι διάσπαρτα σημεία (κορυφές) που ενώνονται με γραμμές (ακμές). Γράφους μπορούμε να συνατήσουμε σε διάφορα προβλήματα της καθημερινότητας όπως δίκτυα υποδομών (πχ δίκτυο ύδρευσης), προβλήματα χαρτογράφησης (πλοήγηση), τηλεπικοινωνιών (πχ δορυφόροι), μεταφορών (πχ σιδηρόδρομοι) , και άλλα \cite{Μανωλόπουλος}. Η θεωρία γράφων είναι ένας σημαντικός τομέας των μαθηματικών γιατί πέρα απ' το γεγονός ότι με χρήση αυτών μπορούμε να μοντελοποιήσουμε εύκολα προβλήματα της καθημερινότητας μας σε τομείς που αναφέρθηκαν παραπάνω, μπορούμε επίσης να αναπτύξουμε αλγόριθμους που λύυνουν προβλήματα με χρήση γράφων. Παράδειγμα αυτού είναι και ο αλγόριθμος αποικίας των μυρμηγκιών που θα αναλύσουμε σε αυτήν την πτυχιακή εργασία.

\subsection{Βασικοί Ορισμοί και έννοιες}
Για να γίνουν κατανοητά όσα θα αναφερθούν στην παρακάτω εργασία είναι απαραίτητο να παρουσιαστεί το θεωρητικό υπόβαθρο πάνω στο οποίο είναι βασισμένοι οι αλγόρθμοι βελτιστοποίησης. Για πιο αναλυτική μελέτη παραπέμπονται οι βιβλιογραφικές αναφορές που χρησιμοποιήθηκαν στο τέλος της εργασίας.

Ένας γράφος είναι μία μαθηματική δομή που ορίζεται με αυστηρό τρόπο μέσω δύο συνόλων: το σύνολο κόμβων (ή κορυφών) και το σύνολο ακμών (ή γραμμών) που συνδέουν ζεύγη κορυφών μεταξύ τους και χρησιμοποιείται για την αναπαράσταση πληροφορίας σχετικά με συνδεσμολογία \cite{Ντενισιώτης}. Όταν δύο κορυφές είναι συνδεδεμένες, δηλαδή ενώνονται με μία ακμή ονομάζονται γειτονικές (για παραδειγμα στο [Σχήμα 5] η κορυφή Α και η κορυφή Β είναι γειτονικές), αντίστοιχα δύο ακμές που καταλήγουν σε ίδια κορυφή ονομάζονται προσπίπτουσες της κορυφής αυτής. Το πόσες ακμές προσπίπτουν σε μία κορυφή είναι ο βαθμός της κορυφής αυτής. Τάξη ενός γράφου καλούμε το πόσες κορυφές έχει (για παράδειγμα στο [Σχήμα 5] η κορυφή Ε είναι 4ου βαθμού και ο γράφος είναι τάξης 6) . Ένας γράφος μπορεί να είναι είτε κατευθυνόμενος [Σχήμα 6] όταν οι ακμές έχουν μια κατεύθυνση από έναν κόμβο προς έναν άλλο, είτε μη-κατευθυνόμενος [Σχήμα 5] όταν οι ακμές δεν έχουν κατεύθυνση και μπορούν να πηγαίνουν προς οπουδήποτε μεταξύ των κόμβων. 

Στις ακμές ενός γράφου μπορούν να επισυναπτούν βάρη, τα οποία αντιπροσωπεύουν το κόστος, την απόσταση ή άλλες χρήσιμες πληροφορίες που συνδέονται με τις σχέσεις μεταξύ των κόμβων. Τα βάρη μπορούν να χρησιμοποιηθούν για την εκτέλεση αλγορίθμων βελτιστοποίησης και την αναζήτηση των βέλτιστων μονοπατιών στον γράφο \cite{Ντενισιώτης}, \cite{Γκέρτσης}. Στον αλγόριθμο που θα μελετήσουμε τα βάρη αντιπροσοπεύουν την απόσταση της διαδρομής ή το επίπεδο της φερομόνης (θα αναλυθεί παρακάτω). 
\begin{figure}[h]
    \begin{minipage}[c]{.46\linewidth}
        \centering
        \includegraphics[scale=0.15]{2947_thesis/pictures/undirected.png}
        \caption{Μη κατευθυνόμενος γράφος.}
    \end{minipage}
    \hfill%
    \begin{minipage}[c]{.46\linewidth}
        \centering
        \includegraphics[scale=0.15]{2947_thesis/pictures/directed.png} 
        \caption{Κατευθυνόμενος γράφος.}
    \end{minipage}
\end{figure}

\subsection{Μαθηματικό υπόβαθρο}
Ένας γράφος $G$ ορίζεται από δύο σύνολα $V$ και $E$. Το σύνολο $V$ είναι ένα πεπερασµένο σύνολο (μη άπειρο), που περιέχει ως στοιχεία τις κορυφές του γράφου. Το σύνολο $E$ περιέχει τις ακμές ενός γράφου εκφρασμένες με δισύνολα δύο γειτονικών κορυφών. Έτσι, πεπερασμένος (μη - κατευθυνόμενος) γράφος, λέγεται το διατεταγμένος ζεύγος $G = (V(G), E(G))$ των πεπερασμένων συνόλων $V(G)$, $E(G)$ \cite{Ντενισιώτης}. Αν πάρουμε ως παράδειγμα το γράφο $G$ στο [Σχήμα 7] παρατήρουμε ότι τα σύνολα $V(G)$ και $E(G)$ έχουν ως εξής: 
\begin{itemize}
  \item $V(G)=$[$v1,v2,v3,v4$]
  \item $E(G)=$[$e1(v1,v2),e2(v1,v3),e3(v2,v3),e4(v3,v4)$] 
\end{itemize}

\begin{figure}
    \centering
    \includegraphics[scale=0.30]{2947_thesis/pictures/synola.png} 
    \caption{Σύνολα γραφήματος.}
\end{figure}

Επομένως ένας γράφος είναι ένα μαθηματικό αντικείμενο που ορίζεται με αυστηρό τρόπο μέσω δύο συνόλων: το σύνολο κόμβων και το σύνολο ακμών. Το σύνολο των ακμών ενος γράφου μπορεί να είναι κενό, αυτό δεν ισχύει όμως για το σύνολο των κορυφών.

\subsection{Αναπαράσταση γράφων}
Την κλασσική μορφή αναπαράστασης ενός γράφου την είδαμε ήδη παραπάνω [Σχήμα 3], όμως μια τέτοια αναπαράσταση δεν είναι καθόλου πρακτική σε προγραμματιστικό επίπεδο. Για αυτό αν θέλουμε να αναπαραστίστουμε γράφους σε έναν υπολογιστή χρησιμοποιούμε δομές δεδομένων. Οι δύο πιο βασικοί μέθοδοι αναπαράστασης γράφων σε υπολογιστές είναι οι πίνακες γειτνίασης και οι λίστες γειτνίασης. 

Πίνακα γειτνίασης ονομάζουμε ένα πίνακα μεγέθους $nxn$, όπου $n$ ο αριθμός των κορυφών του γράφου. Κάθε κελί του πίνακα δείχνει την σχέση των αναγραφώμενων κορυφών. Σε ένα μη-κατευθυνόμενος γράφο το κελί $(i, j)$ περιέχει τον αριθμό των ακμών που συνδέουν τον κόμβο $i$ με τον κόμβο $j$. Σε έναν κατευθυνόμενο γράφο, το κελί $(i, j)$ μπορεί να περιέχει τον αριθμό των ακμών που πηγαίνουν από τον κόμβο $i$ στον κόμβο $j$. Οι πίνακες γειτνίασης μπορεί να έχουν και βάρη που είναι μια επέκταση του απλού πίνακα γειτνίασης, σε αυτήν την πείπτωση, το έκαστο κελί ενός πίνακα περιέχει έναν αριθμό που ονομάζετε βάρος και υποδηλώνει κάτι ανάλογα με την χρήση του, σε έναν αλγόριθμο βελτιστοποίησης το βάρος μπορεί να υποδηλώνει την απόσταση της μιας κορυφής από την άλλη, την πιθανότητα επιλογής αυτής της διαδρομής, την επιρροή που δέχεται κάποια οντότητα σε επόμενο πιθανό πείραμα ή οποιοδήποτε άλλο κριτήριο που ανταποκρίνεται στον σκοπό του συγκεκριμένου αλγορίθμου.

Ας πάρουμε για παράδειγμα τον γράφο στο [Σχήμα 7], πρόκειται για έναν μη-κατευθυνόμενο γράφο χώρις βάρη με 4 κορυφές, δηλαδή $n=4$ και με ακμές που φαίνενται στο σύνολο $E(G)$. Επομένως ο πίνακας γειτνίασης του θα διαμορφώνεται έτσι: 
$$
G_{n,n} = 
 \begin{array}{c|c c c c}
    & v_{1} & v_{2} & v_{3} & v_{4} \\ \hline
    v_{1} & v_{1,1} & v_{1,2} & v_{1,3} & v_{1,4} \\
    v_{2} & v_{2,1} & v_{2,2} &   v_{2,3} & v_{2,4} \\
    v_{3} & v_{3,1} & v_{3,2} & v_{3,3} & v_{3,4} \\
    v_{4} & v_{4,1} & v_{4,2} & v_{4,3} & v_{4,4} 
 \end{array}
 $$
 που με χρήση 0-1 γίνεται έτσι: 
 $$
G_{4,4} = 
 \begin{array}{c|c c c c}
    & v_{1} & v_{2} & v_{3} & v_{4} \\ \hline
    v_{1} & 0 & 1 & 1 & 0 \\
    v_{2} & 1 & 0 & 1 & 0 \\
    v_{3} & 1 & 1 & 0 & 1 \\
    v_{4} & 0 & 0 & 1 & 0 
 \end{array}
 $$
όπου 1 συμβολίζει ότι αυτές οι δύο κορυφές είναι γειτονικές έχοντας ακμή να τις ενώνει, ενώ 0 ότι δεν είναι. Με λίγα λόγια, αν το ζεύγος δύο κορυφών υπάρχει στο σύνολο $E(G)$ τότε το συγκεκριμένο κελί στον πίνακα γειτνίασης θα πάρει την τιμή 1, αλλιώς 0. Σε έναν τέτοιο πίνακα υπάρχει επανάληψη πληροφορίας καθώς το $v(i,j)$ είναι ίδιο με το $v(j,i)$. Σε περίπτωση κατευθυνόμενου γράφου όμως αυτό δε θα συνέβαινε αφού το κελί $v(i,j)$ θα συμβόλιζε αν υπάρχει ακμή από την κορυφή $i$ προς την κορυφή $j$, ενώ το κελί $v(j,i)$ θα συμβόλιζε αν υπάρχει ακμή από την κορυφή $j$ προς την κορυφή $i$. Σε περίπτωση ύπαρξης βαρών, τα κελιά με την τιμή 1 στον πίνακα, αντί για 1, θα είχαν την τιμή του αντίστοιχου βαρους. 

Λίστα γειτνίασης ονομάζουμε μια αναπαράσταση γράφων όπου για κάθε κορυφή διατηρείται μια λίστα των γειτόνων της. Σε περίπτωση κατευθυνόμενου γράφου μπορεί να υπάρχει ξεχωριστή λίστα για τους εξερχόμενους και τους εισερχόμενους γείτονες. Αυτή η αναπαράσταση αποκτά αξία σε γράφους με αραιή συνδεσιμότητα. 

Το πόσο μεγάλη θα είναι η λίστα εξαρτάται από τον αριθμό των κορυφών του γράφου. Κάθε στοιχείο στην λίστα συβολίζει μία ακμή του γράφου.
 
 Σε έναν μη-κατευθυνόμενο γράφο, η λίστα γειτνίασης για κάθε κορυφή περιλαμβάνει τους γείτονές της, δηλαδή τις άλλες κορυφές με τις οποίες συνδέεται με μια ακμή. Αν υπάρχουν $n$ κορυφές στον γράφο, η λίστα γειτνίασης για κάθε κορυφή περιλαμβάνει μια λίστα με το πλήθος των γειτόνων της. Για παράδειγμα στο [Σχήμα 7] που απεικονίζεται ένας μη-κατευθυνόμενος γράφος με 4 κορυφές, άρα $n=4$ και ακμές που φαίνενται στο σύνολο $E(G)$, αν η αναπαράσταση γινόταν με λίστα γειτνίασης θα ήταν ως εξής: 
\begin{itemize}
    \item $v1: \{v2, v3\}$
    \item $v2: \{v1, v3\}$
    \item $v3: \{v1, v2, v4\}$
    \item $v4: \{v3\}$
\end{itemize}

Σε περίπτωση κατευθυνόμενου γράφου σε κάθε κορυφή θα υπάρχαν 2 λίστες, μία που θα εξέφραζε τους προηγούμενες κορυφές και μία που θα εξέφραζε τις ακόλουθες. Για παράδειγμα στο [Σχήμα 6] η κορυφή Γ σε λίστα γειτνίασης θα αναπαριστόταν ως εξής: 
\begin{itemize}
    \item Γ: προγούμενοι:$\{E\}$ ακόλουθοι:$\{A, Z\}$
\end{itemize}

Σε σχέση με τους πίνακες γειτνίασης οι λίστες γειτνίασης επιτρέπουν την εύκολη πρόσβαση στα δεδομένα του γράφου καθώς και τροποποίηση αυτών, όμως η εισαγωγή και η διαγραφή μιας ακμής είναι πιο χρονοβόρα. Επίσης σε γράφους με λίγες ακμές απαιτούν λίγοτερη μνήμη, ενώ σε πλήρη συνδεδεμένους γράφους περισσότερη. 
