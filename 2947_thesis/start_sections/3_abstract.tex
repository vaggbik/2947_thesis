\begin{abstract}
Σκοπός της παρούσας πτυχιακής εργασίας είναι η παρουσίαση και υλοποίηση του αλγόριθμου της αποικίας των μυρμηγκιών. Ο αλγόριθμος αυτός είναι ένας αλγόριθμος βελτιστοποίησης που είναι βασισμένος σε παρατηρήσεις που έγιναν στα μυρμήγκια στην φύση. Πιο συγκεκριμένα, έχει μελετηθεί ο τρόπος με τον οποίο καταφέρνουν να αλληλεπιδρούν μεταξύ τους και το πως βρίσκουν πάντα την βέλτιστη διαδρομή από την φωλιά τους μέχρι την πηγή τροφής. 

Αυτό το πετυχαίνουν εκπέμπτοντας χημικές ουσίες που τις χρησιμοποιούν ως σήματα επικοινωνίας, τις λεγόμενες φερομόνες. Ο αλγόριθμος που θα αναλύσουμε είναι μία προσομοίωση αυτής της συμπεριφοράς με χρήση τεχνητών μυρμηγκιών.
 
Το γεγονός ότι βρίσκει βέλτιστη λύση σε συνδυασμό με το πόσο γρήγορα επιτυγχάνεται και το ότι μπορούν να εφαρμοστούν πολλές παραλλαγές του ανάλογα με το ζητούμενο καθιστά αυτόν τον αλγόριθμο χρήσιμο σε διάφορους τομείς, όπως η επεξεργασία εικόνων και οι τηλεπικοινωνίες. Εντάσεται σε πολλά πεδία και εφαρμογές που έχουν ως στόχο την βελτιστοποίηση, με κλασικό παράδειγμα αυτό του πλανόδιου πωλητή.

Παρακάτω, θα αναλυθεί η θεωρία γράφων και το πως σχετίζεται με αλγόριθμους βελτιστοποίησης, θα αναφερθούν κάποιοι αλγόριθμοι βελτιστοποίησης με έμφαση στον αλγόριθμο της αποικίας των μυρμηγκιών. Θα παρουσιαστεί η δική μου υλοποίηση, με διάφορα παραδείγματα εκτέλεσης του και θα δωθούν πιθανές εφαρμογές σε προβλήματα πραγματικόύ κόσμου σε τομείς όπως η επιστήμη των υπολογιστών. Όλοι οι αλγόριθμοι θα υλοποιηθούν σε python.

{\bf Λέξεις Κλειδιά}.
    Προβλήματα Βελτιστοποίησης,  Αλγόριθμος Αποικίας Μυρμηγκιών, Θεωρία Γράγων {\lt, Python} 
\end{abstract}    
\newpage
{\latintext
\begin{center}
{\sc{summary}}
\end{center}
{\small
	{\lt The purpose of this thesis is the presentation and implementation of the ant colony algorithm. This algorithm is an optimization algorithm that is based on observations made on ants in nature. More specifically, the way in which they manage to interact with each other and how they always find the optimal route from their nest to the food source has been studied.

  They achieve this by emitting chemicals that they use as communication signals, so-called pheromones. The algorithm we will analyze is a simulation of this behavior using artificial ants.
 
 The fact that it finds an optimal solution combined with how fast it is achieved and that many variations of it can be applied depending on the application make this algorithm useful in various fields such as image processing and telecommunications, and span many fields and applications. which aim at optimization, with the classic example of that of the traveling salesman.

Below we will discuss graph theory and how it relates to optimization algorithms, introduce some optimization algorithms with an emphasis on the ant colony algorithm, present my own implementation and compare it to variations of this and other optimization algorithms, and give possible applications in real-world problems in fields such as computer science. All algorithms will be implemented in python.}
	\ \\\\
{\bf Key Words.} Οptimization Problems, Ant Colony Algorithm, Graph Theory, Python
}}